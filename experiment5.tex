\section{Exp5: Pushing task with pose estimation from LIDAR}

This aim of this experiment was originally to show that a policy could be
learned on real robots for pushing a cube to some target pose. However,
experiments in simulation were harder than anticipated and consumed several
weeks before solving the problem. Also, when adding the same amount of noise
expected to come from sensors to the simulation problem, it failed to converge
to any decent policy. The policy trained in the non-noisy simulation worked
quite well when applied to the real setup which is described in more detail
below.

\subsection{Pose estimation}

To estimate the pose of the cube, a \textit{LIght Detection And Range} (LIDAR)
was used by placing it in front of the robot arm (figure \ref{fig:eef-frame}).
The LIDAR returns $360$ distance measurements at approximately evenly spaced
angles which vary from scan to scan. One series of scans is returned at $10$
Hz. The cube is a red wooden cube with all sides measuring $4$ cm. One of these
sides can be found from the LIDAR-measurements by plotting the points onto a
matrix/image and using the Hough transform \cite{duda1972use}. The Hough
transform returns the equations for the lines that make up the sides of the
cube, but not where the sides start and end. Once the lines are known, the
starts and ends can easily be inferred from the scans.  Knowing at least one
side of the cube, the rotation and position of the cube can also be inferred.
For the following experiments, only the position of the cube was kept, ignoring
the rotation. Conversion of the found position of the cube to robot-frame was
done using a least-squares approach after randomly placing the cube at several
positions known from the forward-kinematics of the arm.

\begin{figure}[h!]
    \centering
    \includegraphics[width=0.48 \textwidth]{res/camera_placement_fixed.jpg}

    \caption{LIDAR seen on the right side in the picture measuring positions of
    the cube. The camera is not used in this experiment.}

    \label{fig:eef-frame}
    
\end{figure}

\subsection{Results}

The policy trained in simulation successfully pushes the cube towards a goal
set in the center of the workspace, video of this can be seen here:
\url{https://youtu.be/82XNDBPbJH0}. Sometimes the policy fails to push the cube
further due to small action commands not affecting the robot, and due to
differences in the shapes of the simulation object (circle) and the cube.
Adding a small noise term alleviated these problems and was used in the linked
video.
