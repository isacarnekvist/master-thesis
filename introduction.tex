\section{Introduction}

\subsection{Robotic manipulation}

Reinforcement Learning (RL) deals with learning what actions to do in order to
reach a long-term goal (a \textit{policy} is learned) and have been widely used
for learning robotic manipulation tasks. Recent research suggests methods
capable of learning real-world robotic manipulation tasks without simulation
pre-training, learning only from real-world experience
\cite{yahya2016collective,gu2016deep,finn2016deep,chebotar2016path}. Using
visual feedback for manipulation tasks is a way to handle unknown poses of
manipulators and target objects and training these kind of tasks end-to-end
have been successful in simulation tasks
\cite{schulman2015trust,lillicrap2015continuous}. Real-world manipulation tasks
have also been learned with visual feedback but using pre-trained pose
estimation networks \cite{gu2016deep}. Pose estimation have been shown to work
by using convolutional neural networks (CNN)
\cite{levine2016end,chebotar2016path,yahya2016collective}, although for some
cases it was shown that test-time translations severely effected manipulation
task performance \cite{yahya2016collective}. CNNs have been trained to deal
with relative poses \cite{park20163d}, and this could be a feasible solution to
deal with unknown random camera offsets. Real-world experiments often rely on
human demonstrations to learn a successful policy but this might not always be
available. Recently a successful demonstration of learning a door opening task
from scratch without the need for human demonstration or simulation
pre-training have been shown \cite{gu2016deep}. This was using a version of
Normalized Advantage Function algorithms (NAF) \cite{gu2016continuous}
distributed over several robotic platforms.

\subsection{This study}

A series of experiments were conducted in order to verify the distributed
version of NAF in new contexts. Relatively inexpensive robotic
cartesian-controlled arms were used for pushing objects into randomly sampled
poses. To further extend previous results \cite{gu2016deep}, pose estimation
was done using a CNN in accordance with previous work
\cite{levine2016end,chebotar2016path,yahya2016collective}. It was further
evaluated whether training the pose estimation network with randomly placed
camera positions could make it more robust to translations.

\subsection{Motivation}

Positive results could imply further automation of manipulation tasks where
relative poses of robots, sensors, and targets are unknown. Using a CNN that
estimates relative poses and that is robust to translations could be
interesting also for researchers and industry beyond the robotics use case as
in this thesis.
