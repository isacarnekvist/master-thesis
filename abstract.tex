\begin{abstract}

Reinforcement learning was recently successfully used for real-world robotic
manipulation tasks, without the need for human demonstration, using a
\textit{normalized advantage function}-algorithm (NAF). Limitations on the
shape of the advantage function however poses doubts to what kind of policies
can be learned using this method. For similar tasks, convolutional neural
networks have been used for pose estimation from images taken with
fixed-position cameras. For some applications however, this might not be a
valid assumption. It was also shown that the quality of policies for robotic
tasks severely deteriorates from small camera offsets. NAF is in this thesis
successfully used to solve simple tasks on real robotic systems where data
collection is distributed over several robots, and learning is done on a
separate server.  Using NAF to learn a pushing task, with clear multi-modal
properties, however fails to converge to a good policy, both on the real robots
and in simulation. Deep deterministic policy gradient (DDPG) is instead used
in simulation and successfully learns to solve the task. The learned policy is
then applied on the real robots and accomplishes to solve the task in the real
setting as well. Pose estimation from fixed position camera images is learned
and the policy is still able to solve the task using these estimates. By
defining a coordinate frame from an object visible to the camera, in this case
the robot arm, a neural network learns to regress the pushable objects pose in
this frame without the assumption of a fixed camera. However, the precision of
the predictions were too inaccurate to be used for solving the pushing task.
Further modifications to this approach could however show to be a feasible
solution to randomly placed cameras with unknown poses.

\end{abstract}


\begin{otherlanguage}{swedish}
    \begin{abstract}

        Reinforcement learning har nyligen använts framgångsrikt för att lära
        icke-simulerade robotar uppgifter med hjälp av en \textit{normalized
        advantage function}-algoritm (NAF), detta utan att använda mänskliga
        demonstrationer.  Restriktioner på funktionsytorna som använts kan dock
        visa sig vara problematiska för generalisering till andra uppgifter.
        För pose-estimering har i liknande sammanhang convolutional neural
        networks använts med bilder från kamera med konstant position. I vissa
        applikationer kan dock inte kameran garanteras hålla en konstant
        position och studier har visat att kvaliteten på policys kraftigt
        förvärras när kameran förflyttas. NAF appliceras i denna uppsats
        framgångsrikt på enklare problem där datainsamling är distribuerad över
        flera robotar och inlärning sker på en central server. Vid applicering
        på en uppgift med tydliga multimodala egenskaper misslyckas dock NAF.
        Deep deterministic policy gradient (DDPG) appliceras istället på
        problemet och lär sig framgångsrikt att lösa problemet i simulering.
        Den inlärda policyn appliceras sedan framgångsrikt på riktiga robotar.
        Pose-estimering genom att använda en fast kamera implementeras också
        framgångsrikt. Genom att definiera ett koordinatsystem från ett föremål
        i bilden med känd position, i detta fall robotarmen, kan andra föremåls
        positioner beskrivas i denna koordinatram med hjälp av neurala nätverk.
        Dock så visar sig precisionen vara för låg för att appliceras på
        robotar. Resultaten visar ändå att denna metod, med ytterligare
        utökningar och modifikationer, skulle kunna lösa problemet.

    \end{abstract}
\end{otherlanguage}
