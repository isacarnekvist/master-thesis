\chapter{Conclusions and future work}

In this thesis, NAF was evaluated in both a simulated setting and distributed
over several robots. Initial experiments with simple reaching tasks worked
fine, but extending this to a pushing task failed, even after simplifications
like keeping starting and goal positions fixed. This could either be due to
some error in the implementation that did not affect the simpler reaching task,
or that the restriction to a quadratic advantage function could not represent
the bi-modal nature of the task. DDPG however could solve this task but did not
prove stable under noisy controls and observations. Reformulation of the
problem as a partially observable Markov decision process might be an
interesting way forward since highly accurate sensors are not always what are
seen in reality.

Regarding pose estimation, a network was proposed to incorporate depth
information, a common feature nowadays even in consumer products. Simulation
showed better results when adding this feature, although for real data no
benefit was seen. Future experiments could include replacing zero representing
missing depth values with something else, for example the maximum distance
registered. Relative pose estimation was clearly shown to be capable of
learning the geometric interpretation from 2D data, although the precision was
too low for this use case. This could have been due to the neural network not
being able to represent this accurately enough, the resolution of input data or
intermediary representations being too low, or labels during training being too
noisy requiring more data for higher quality predictions. The former case is
somewhat supported by the case that simulated data giving 2D coordinates also
produced noisy estimates.
