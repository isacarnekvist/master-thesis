\section{Background}

Recently a number of papers were published were robots learn manipulation tasks
through shared experiences
\cite{gu2016deep,finn2016deep,yahya2016collective,chebotar2016path}. It would
be interesting to see whether these methods can be used and altered to
generalize for other contexts such as changing the type of manipulators and
performing different tasks. The use cases are for robotic manipulation with
camera as feedback where exact relative positions of objects, manipulators, and
sensors need not be fixed, also where resources exists to use several robots
for speeding up the learning process. The experiments needed for this thesis would be held at the
Robotics, Perception, and Learning lab at KTH.

This study would mainly require choice, implementation and scientific
evaluation of parallelized machine learning algorithms, coding for robotic
manipulation, and ground truth generation. The machine learning algorithms
would include both supervised learning for computer vision and reinforcement
learning for performing tasks. This makes the project not only suitable for a
computer science degree project, but also for the master program in machine
learning.

\subsection{Research question}

How can deep reinforcement learning be used for performing manipulation where
multiple robots are used in parallel for training?

\section{Method}

A manipulation task would be set up with multiple cameras, computers, and
robotic arms. Building on the ideas mainly from the articles
\cite{gu2016deep,finn2016deep,yahya2016collective,chebotar2016path} I would try
out this with robots that are different in the way their motion is controlled
(torque vs. cartesian). The algoritms for controlling the arm would be
end-to-end, i.e. that the entire pipeline from camera data to control signals
to the arm would be learned.

\section{Limitations}

A first part of the work would to pre-train a part of the neural network for
pose estimation (and training data generation). If this turns out to need
a lot of time, the method for accomplishing the manipulation task might be
chosen to better match the remaining time.


\section{Students background}

I have a bachelor's degree in computer science and finishing the machine
learning master program. I have taken both first cycle and second cycle
statistics courses, several machine learning courses including computer vision
and several kinds of supervised- and unsupervised learning. I have experience
with implementing convolutional and regular neural networks from courses and
from working. I have also recently finished a course in robotics and autonomous
systems. I have no courses remaining to be finished except those that are
planned until the start of the thesis project.

The courses that are ongoing and required for graduating are:
\begin{itemize}
    \item Introduction to the Philosophy of Science and Research Methodology (DA2205)
    \item Program Integrating Course in Machine Learning (DD2301)
    \item Robotics and Autonomous Systems (DD2425)
    \item Statistical Methods in Applied Computer Science (DD2447)
\end{itemize}
