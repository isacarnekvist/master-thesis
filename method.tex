\section{Method}

\subsection{Robotic arms}

The manipulation task to be learned is to push an object from and to random
positions within the workspace. The robotic arms used for this were relatively
cheap arms with 4 degrees of freedom, whereof only 3 degrees were used. These
could be controlled by either using a cartesian controller, or directly sending
commands to the servos. They do not support torque control. The calibration
tools and controllers for these robots did not work very well, so software for
this was implemented including derivation of inverse and forward kinematics.

\subsection{Software and libraries}

For high-level computational graphs, Keras \cite{chollet2015keras} was used.
For lower-level extensions not supported by Keras, Theano was used
\cite{theano2016theano}. For image pre-processing, plotting, and overall linear
algebra and mathematical computations, SciPy was used \cite{scipy2016scipy}.
