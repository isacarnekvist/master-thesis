\section{Method}

\subsection{Examination method}

Preliminary method is using the mentioned distributed version of NAF and extend
it with pose estimates from a convolutional neural network. This network is
pretrained as in \cite{yahya2016collective} by randomly placing objects and the
end-effector and this way generating training data. Several robots will be used
to parallelize the training process. The exact task is not set yet but should be
of the kind that needs continuous re-evaluations and decisions until it is
considered to be done.

\subsection{Conditions}

There will be need for several robot setups, each including a robot,
computer, and camera. These will have to be able to communicate with a separate
computer responsible for training the policies/neural networks. In the ideal
case, this computer is supplied and has a graphics card compatible with modern
neural network libraries.

\subsection{Limitations}

A proof of concept should be done with a corresponding report (the thesis).
There are no requirements for implementation of code that can be generalized
and reused in the form of libraries etc. The main contribution in terms of
generalization should be attainable from the thesis. The code will be openly
published on GitHub.
